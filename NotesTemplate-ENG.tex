% !TEX TS-program = pdflatex
% !TEX encoding = UTF-8 Unicode

% TeX-ENG (version 1.0b)
% For Engineering classes at the University of Texast at Austin
% Notes template created by Abdon Morales for the College of Natural Science, 
% for the Department of Mathematics, Cockrell School of Engineering, and the Dept.
% of Computer Science.

% (c) 2024 Abdon Morales and the University of Texas at Austin
% This is a notes template for a LaTeX document using the "article" class for Mathematics (Calculus)
% at the University of Texas at Austin.

% Last change made: Feb 8, 2024 8:32 AM CST


% See "book", "report", "letter" for other types of document.

\documentclass[11pt]{article}

% Package for handling images
\usepackage{graphicx}

% Packages for mathematics
\usepackage{amsmath, amssymb, amsthm}

% Better handling of hyperlinks
\usepackage{hyperref}

% Package for code snippets
\usepackage{listings}

% Setting margins
\usepackage[margin=1in]{geometry}

% Header and Footer Stuff
\usepackage{fancyhdr}
\pagestyle{fancy}
\fancyhead[L]{\slshape \MakeUppercase{Engineering Notes}}
\fancyhead[R]{\slshape{Your Name}}
\fancyfoot[C]{\thepage}

% Title, author, and date
\title{Engineering Notes}
\author{Your Name}
\date{\today}

% Start of the document
\begin{document}
\maketitle
\tableofcontents
\newpage

\section{Introduction}
Briefly introduce the topic of your notes here. Mention the key concepts and the scope of your study or discussion.

\section{Theory}
Discuss the theoretical background relevant to the topic. Use subsections to organize different theories or concepts.

\subsection{Concept 1}
Detail about Concept 1 here.

\subsection{Concept 2}
Detail about Concept 2 here.

\section{Equations}
List down important equations. Use the \texttt{equation} environment to properly format them.

\begin{equation}
E = mc^2
\end{equation}

\section{Examples}
Provide examples to illustrate how theories and equations are applied. You can use the \texttt{enumerate} or \texttt{itemize} environments for bullet points or numbered lists.

\begin{enumerate}
    \item First example...
    \item Second example...
\end{enumerate}

\section{Problems}
Pose problems or questions that are yet to be solved or are for practice.

\begin{enumerate}
    \item First problem...
    \item Second problem...
\end{enumerate}

\section{Conclusion}
Summarize the key points discussed in your notes and any conclusions you have drawn.

% End of the document
\end{document}